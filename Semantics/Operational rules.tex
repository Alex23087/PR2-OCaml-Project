\documentclass{article}
\title{Regole di semantica operazionale per le operazioni su insiemi}
\date{}
\usepackage{mathtools}
\usepackage{amssymb}
\usepackage{empheq}
\usepackage{proof}

\begin{document}
\maketitle
\pagenumbering{gobble}
\section*{EmptySet}
\begin{equation*}
	\infer{EmptySet(T) \Rightarrow SetT(T, \varnothing)}{-}
\end{equation*}
\section*{SingletonSet}
\begin{equation*}
	\infer{SingletonSet(T, e) \Rightarrow SetT(T, \{v\})}{e \Rightarrow v:T}
\end{equation*}
\section*{SetOf}
\begin{equation*}
	\infer{SetOf(T, \{e_1...e_n\}) \Rightarrow SetT(T, \{v_1...v_n\})}{e_1 \Rightarrow v_1:T & ... & e_n \Rightarrow v_n:T}
\end{equation*}
\section*{SetPut}
\begin{equation*}
	\infer{SetPut(e_1, e_2) \Rightarrow SetT(T, s \cup \{v\})}{e_1 \Rightarrow SetT(T, s) & e_2 \Rightarrow v:T}
\end{equation*}
\section*{SetRemove}
\begin{equation*}
	\infer{SetRemove(e_1, e_2) \Rightarrow SetT(T, s \setminus \{v\})}{e_1 \Rightarrow SetT(T, s) & e_2 \Rightarrow v:T}
\end{equation*}
\section*{SetIsEmpty}
\begin{equation*}
	\infer{SetIsEmpty(e) \Rightarrow Bool(s = \varnothing)}{e \Rightarrow SetT(T, s)}
\end{equation*}
\section*{SetContains}
\begin{equation*}
	\infer{SetContains(e_1, e_2) \Rightarrow Bool(v \in s)}{e_1 \Rightarrow SetT(T, s) & e_2 \Rightarrow v:T}
\end{equation*}
\section*{SetIsSubset}
\begin{equation*}
	\infer{SetIsSubset(e_1, e_2) \Rightarrow Bool(s_1 \subseteq s_2)}{e_1 \Rightarrow SetT(T, s_1) & e_2 \Rightarrow SetT(T, s_2)}
\end{equation*}
\section*{SetMin}
\begin{equation*}
	\infer{SetMin(e) \Rightarrow min(s):T}{e \Rightarrow SetT(T, s) & T \neq Closure(t,r)}
\end{equation*}
\section*{SetMax}
\begin{equation*}
	\infer{SetMax(e) \Rightarrow max(s):T}{e \Rightarrow SetT(T, s) & T \neq Closure(t,r)}
\end{equation*}
\section*{ForAll}
\begin{equation*}
	\infer{Forall(e_1, e_2) \Rightarrow Bool(\forall x \in s.\; Apply(f, x) \Rightarrow Bool(true))}{e_1 \Rightarrow SetT(T, s) & e_2 \Rightarrow f:T \rightarrow Boolean}
\end{equation*}
\section*{Exists}
\begin{equation*}
	\infer{Exists(e_1, e_2) \Rightarrow Bool(\exists x \in s.\; Apply(f, x) \Rightarrow Bool(true))}{e_1 \Rightarrow SetT(T, s) & e_2 \Rightarrow f:T \rightarrow Boolean}
\end{equation*}
\section*{Filter}
\begin{equation*}
	\infer{Filter(e_1, e_2) \Rightarrow SetT(T, \{x \in s : Apply(f, x) \Rightarrow Bool(true)\})}{e_1 \Rightarrow SetT(T, s) & e_2 \Rightarrow f:T \rightarrow Boolean}
\end{equation*}
\section*{Map}
\begin{equation*}
	\infer{Map(e_1, e_2) \Rightarrow SetT(T_2, \{Apply(f,x \in s)\})}{e_1 \Rightarrow SetT(T_1, s) & e_2 \Rightarrow f:T_1 \rightarrow T_2}
\end{equation*}
\section*{SetUnion}
\begin{equation*}
	\infer{SetUnion(e_1, e_2) \Rightarrow SetT(T, s_1 \cup s_2)}{e_1 \Rightarrow SetT(T, s_1) & e_2 \Rightarrow SetT(T, s_2)}
\end{equation*}
\section*{SetIntersection}
\begin{equation*}
	\infer{SetIntersection(e_1, e_2) \Rightarrow SetT(T, s_1 \cap s_2)}{e_1 \Rightarrow SetT(T, s_1) & e_2 \Rightarrow SetT(T, s_2)}
\end{equation*}
\section*{SetSubtraction}
\begin{equation*}
	\infer{SetSubtraction(e_1, e_2) \Rightarrow SetT(T, s_1 \setminus s_2)}{e_1 \Rightarrow SetT(T, s_1) & e_2 \Rightarrow SetT(T, s_2)}
\end{equation*}
\end{document}
